\documentclass{article}
\usepackage[utf8]{inputenc}
\usepackage{amsmath}

\title{Homework \#2}
\author{Simon Judd}
\date{\today}

\begin{document}

\maketitle

\section*{Problem 1. (Sumcheck with tensor weights)}

We consider an extension of the sumcheck problem where the summand is multiplied by weights that have a product structure. Specifically, given $n \cdot |H|$ field elements $\{\delta_{i,\alpha} \in F\}_{i\in[n],\alpha\in H}$, we consider statements of the following form:

\[
\sum_{\alpha_1,\ldots,\alpha_n \in H} \delta_{1,\alpha_1} \cdots \delta_{n,\alpha_n} \cdot p(\alpha_1, \ldots, \alpha_n) = \gamma
\]

Show that the sumcheck protocol can be extended to support the above statement, with the same completeness and soundness guarantees.


\subsection{Proof}

The key here is to represent $\delta_{n, \alpha_{n}}$ as a low-degree polynomia l.  For each $i \in [n]$ we construct a low-degree polynomial $\delta_{i}(x)$ over $H$.  The size of this polynomial is bounded by $|H|$.  We construct this polynomial using langrage interpolation:

$$
\delta_{i}(x) = \sum_{\alpha \in H} \delta_{i, \alpha} \cdot L_{a}(x)
$$

Where $L_{a}(x)$ is the lagrange basis polynomial for all $\alpha \in H$.  This ensures that $\delta_{i} = \delta_{i, \alpha}$ for all $\alpha \in H$.

Next we define a new function $f$:


$$
f(x_{1}, .., x_{n}) = (\prod_{i=1}^{n} \delta_{i}(x_{i})) \cdot p(x_{1},..,x_{n})
$$

This is a polynomial in $n$ variables with a degree in each $x_{1}$ of:

$$
deg_{xi}(f) = deg(\delta_{i}) + deg_{xi}(p) \le |H| + deg_{xi}(p)
$$

We now apply the sumcheck protocol to $f$.

For $n \in N$ rounds and for a claimed sum $\gamma$, we define the sum-check protocol as:

$$
\sum_{\alpha_{1},..,\alpha_{n}\in H} f(\alpha_{1},..,\alpha_{n}) = \gamma
$$

In each $1,..n$-rounds, the prover first calculates a polynomial $f_{1}(x)= \sum_{\alpha_{2},..,\alpha_{n}}f(x, \alpha_{2},..,\alpha_{{n}})$ and then sends the verifier $f_{1} \in F[X]$. The verifier checks whether $\sum_{\alpha_{1}\in H}f_{1}(\alpha_{1})=\gamma$ and then responds with a random $w_{1} \in F$.


In the subsequent rounds, the prover sends the following polynomial to the verifer.

$$
f_{2}(x)= \sum_{\alpha_{3},..,\alpha_{n}}f(w_{1}, x ,\alpha_{3},..,\alpha_{{n}})
$$

The verifier then checks $\sum_{\alpha_{2}\in H}f_{2}(\alpha_{2}) = f_{1}(w_{1})$.  Finally at the end of the protocol, the verifier checks whether $f(w_{1},..,w_{n}) = f_{n}(w_{n})$.

\begin{itemize}
        \item Completeness: If the prover is honest and the claimed sum
$\gamma$ is correct, the protocol will accept with probability 1.
        \item Soundness: If the prover is dishonest or $\gamma$ is incorrect, the protocol will reject with high probability, depending on the field size and the degrees of the polynomials involved.
\end{itemize}

\section*{Problem 2. (Efficient multilinear extension)}

The multilinear extension of a boolean function $f : \{0, 1\}^n \rightarrow \{0, 1\}$ over a field $F$ is the unique multilinear polynomial $\mathrm{MLE}_F(f) \in F[X_1, \ldots, X_n]$ that agrees with $f$ on $\{0, 1\}^n$:

\[
\mathrm{MLE}_F(f)(X_1, \ldots, X_n) := \sum_{b_1,\ldots,b_n \in \{0,1\}} f(b_1, \ldots, b_n) \prod_{i\in[n]:b_i=1} X_i \prod_{i\in[n]:b_i=0} (1 - X_i)
\]

Prove that evaluating a multilinear extension at a single point is in linear time. Namely, give an algorithm that given a boolean function $f$ (represented as a string of $2^n$ bits), finite field $F$, and evaluation point $(\alpha_1, \ldots, \alpha_n) \in F^n$, computes the evaluation of $\mathrm{MLE}_F(f)$ at $(\alpha_1, \ldots, \alpha_n)$ in $O(2^n)$ field operations.

Hint: The multilinear extension can be evaluated by summing term by term in $O(n \cdot 2^n)$ field operations while maintaining a state of $O(1)$ field elements in memory. How can you use more memory to speed up the computation?


\subsection{Proof}
The core idea here is that evaluating a multilinear extension is like traversing through the layers of a multidimensional hypercube.

Given a point $\alpha_{1},\dotsc,\alpha_{n}$ where $\alpha_{i} \in \{0,1\}$, we evaluate each round of the multilinear extension as:
\[
f(\alpha) = \sum_{S \subseteq N} f(1_S) \prod_{i \in S} \alpha_i \prod_{j \not\in S} (1 - \alpha_j)
\]
This requires iterating over all $|H|$ subsets, leading to $O(2^{n})$ complexity. However, we can approach this problem recursively and run much more efficiently. The key thing to understand here is that each layer in a multilinear extension can be evaluated in linear time. Each linear interpolation follows the same formula $f(x) = a(1- a) + b\cdot x$ and requires only $O(n)$-work plus $O(1)$ field element for memory. Essentially, for each new variable or dimension, you are only adding one more layer of work.

Let's walk through the recursion. For the base case where $n=1$, we have:
\[
f(x_{1}) = f(0) \cdot (1 - x_{1}) + f(1) \cdot x_{1}
\]
And for larger values in $n$, we have:
\[
  \mathrm{MLE}_F(f)(x_{1},\dotsc,x_{n}) = (1-x_{n}) \cdot \mathrm{MLE}_F(f_{x_{n}=0})(x_{1}, ..,x_{n-1}) + x_{n} \cdot \mathrm{MLE}_F(f_{x_{n}=1}) (x_{1},.., x_{n-1})
\]
where $f_{x_{n}=0}$ is the multilinear extension over $n$-variables when $x_{n}=0$, and $f_{x_{n}=1}$ is the multilinear extension over $n$-variables when $x_{n}=1$.

Now, let's look at how we can use memory to speed up this computation. Let's create a lookup function to store the results of each subproblem $cache(x_{1}, x_{2},\dotsc,x_{k})$ where $k \le n$.

\[
f(x_{1}, x_{2},\dotsc,x_{n}) = \begin{cases}
  cache(x_{1},\dotsc,x_{n}), & \text{if cached} \\
  (1-x_{n}) \cdot \mathrm{MLE}_F(f_{x_{n}=0})(x_{1}, ..,x_{n-1}) + x_{n} \cdot \mathrm{MLE}_F(f_{x_{n}=1}) (x_{1},.., x_{n-1}), & \text{otherwise}
\end{cases}
\]
Once $f(x_{1},\dotsc,x_{n})$ is computed, it's cached for future use:
\[
cache(x_{1},\dotsc,x_{n}) = f(x_{1},\dotsc,x_{n})
\]
This incurs a memory-space cost of $O(2^{n})$, but reduces the runtime.

\section*{Problem 3. (Efficient sumcheck)}

We analyze the running time of the honest prover in the sumcheck protocol, when proving statements of the form $\sum_{\alpha_1,\ldots,\alpha_n \in H} p(\alpha_1, \ldots, \alpha_n) = \gamma$.

\begin{enumerate}
    \item Prove that the honest prover can be realized in $O(d \cdot |H|^n \cdot |p|)$ operations, where $d$ is the individual degree of $p$ and $|p|$ is the number of operations to evaluate $p$ at any point in $F^n$.
    \item Consider the special case where $H = \{0, 1\}$ and $p$ is multilinear. Prove that, if $p$ is specified via its evaluations on $\{0, 1\}^n$, the honest prover can be realized in $O(2^n)$ field operations.
\end{enumerate}

\subsection{Proof}

For each round, the prover calculates an $n$-variate polynomial $p$ of degree $d$ and evaluates it on $|H|^{n}$ total boolean inputs.  The cost of each evaluation is $|p|$.  And this is repeated for each round.

A polynomial is linear if it has a degree of at most $d=1$.  The cost to evaluate a linear polynomial is $|p|=1$.  Thus the total prover cost for evaluating a linear polynomial is $O(1 \cdot 2^{n} \cdot 1)$ field operations. Which simplifies to $O(2^{n})$.


\section*{Problem 4. (Simple Formulas)}

We say that a fully quantified boolean formula is simple if every occurrence of every variable is separated from its quantification point by at most one universal quantifier ($\forall$) and arbitrarily many other symbols.

\begin{itemize}
    \item This is a simple formula: $\forall x_1 \forall x_2 \exists x_3 ((x_1 \vee x_2) \wedge x_3)$.\\
    What is its value?

First we re-write it this arithmetically as:

        $$
        A = \prod_{x_1 \in \{0,1\}} \prod_{x_2 \in \{0,1\}}\sum_{x_3 \in \{0,1\}} ((x_{1} + x_{2}) \cdot x_{3})
        $$

Evaluate the sum over $x_{3}$

        When $x_{3} = 0: ((x_{1} + x_2) \cdot 0) = 0$

        When $x_{3} = 1: ((x_{1} + x_2) \cdot 1) = x_1 + x_2$

This simplifies to:

        $$A = \prod_{x_1 \in \{0,1\}} \prod_{x_2 \in \{0,1\}} (x_1 + x_2)$$

        Which evaluates to:

        $x_{1}=0, x_{2}=0: 0 + 0 =0$

        $x_{1}=1, x_{2}=0: 1 + 0 =1$

        $x_{1}=0, x_{2}=1: 0 + 1 =1$

        $x_{1}=1, x_{2}=1: 1 + 1 =2$

The product of all these terms is:

   $$A = 0 \cdot 1 \cdot 1 \cdot 2 = 0$$

    \item This formula is not simple: $\exists x_1 \forall x_2 ((x_1 \vee x_2) \wedge \forall x_3 (x_1 \vee x_3))$.\\
    What is its value?

First we re-write it this arithmetically as:

        $$
        A = \sum_{x_1 \in \{0,1\}} \prod_{x_2 \in \{0,1\}}((x_{1} + x_{2}) \cdot \prod_{x_3 \in \{0,1\}}(x_{1} + x_{3}))
        $$

Evaluate the sum over $x_{1}$

        When $x_{1} = 0:(0 + x_{2}) \cdot (0 + x_{3})= x_{2}x_{3}$

        When $x_{1} = 1:(1 + x_{2}) \cdot (1 + x_{3})=1 + x_{2} + x_{3} + x_{2}x_{3}$

        Which simplifies to $1 + x_{2} + x_{3} + 2x_{2}x_{3}$

        Now lets evalaute for each possible value:

        $x_{2}=0, x_{3}=0 : 1 + 0 + 0 + 0 = 1$

        $x_{2}=1, x_{3}=0 : 1 + 1 + 0 + 0 = 2$

        $x_{2}=0, x_{3}=1 : 1 + 0 + 1 + 0 = 2$

        $x_{2}=1, x_{3}=1 : 1 + 1 + 1 + 2 = 4$

        Which sums to $A=9$.


\end{itemize}

We denote by TSQBF language obtained by considering only fully quantified boolean formulas that are simple.
\section*{Problem 5.}

Which of the following fully quantified boolean formulas are simple? What is their value?

\begin{enumerate}
    \item $\forall x_1 \forall x_2 \exists x_3 ((x_1 \wedge x_2 \wedge x_3) \wedge \forall x_4 (\neg x_1 \wedge x_4))$

        Simple.
        $$
        A = \prod_{x_{1} \in \{0,1\}} \prod_{x_{2} \in \{0,1\}} \sum_{x_{3} \in \{0,1\}}((x_{1} \cdot x_{2} \cdot x_{3}) \cdot \prod_{x_{4} \in \{0,1\}}((1-x_{1} \cdot x_{4}))
        $$

        If $x_{1}, x_{2}, x_{4} = 0$ then $A=0$.

        Which enables us to simplify to:

        $$
        A = ((1 \cdot 1 \cdot x_{2}) \cdot ((1 - 1 \cdot 1)))
        $$

        Resulting in a value of $A=0$

    \item $\forall x_1 \forall x_2 \exists x_3 ((x_1 \vee x_3) \wedge \forall x_4 (x_3 \vee (x_2 \wedge x_4)))$

        Not Simple

        Value = 1

    \item $\forall x_1 (\exists x_2 \forall x_3 (x_1 \vee x_2 \vee \neg x_3) \wedge \forall x_4 (\neg x_1 \wedge x_4))$

        Simple

        Value = 0

    \item $\forall x_1 (\exists x_2 \forall x_3 (x_1 \vee x_2 \vee \neg x_3) \wedge \exists x_4 (\neg x_1 \wedge x_4))$

        Simple

        Value = 0


\end{enumerate}

\section*{Problem 6.}

In this question we prove that a fully quantified boolean formula can be efficiently transformed into a simple fully quantified boolean formula with the same value. The general idea is to define a fresh variable for each occurrence of each variable in the original formula.

Let $\Phi$ be a fully quantified boolean formula with variables $x_1, \ldots, x_n$. We define a new formula $\Psi$ that has a variable for each universal quantifier crossed by each variable in $\Phi$. For example, if $x_1$ crosses $k$ universal quantifiers in $\Phi$, then $\Psi$ has variables $x_{1,1}, \ldots, x_{1,k}$.

\begin{enumerate}
    \item Give a boolean formula which is true if and only if $x_1$ and $x_2$ are equal.
    \item Let $\Phi = \exists x_1 \forall x_2 ((x_1 \vee x_2) \wedge \forall x_3 (x_1 \vee x_3))$. By replacing the two occurrences of $x_1$ with $x_{1,1}$ and $x_{1,2}$, and adding constraints and quantifiers, obtain a simple formula $\Psi$ that has the same value as $\Phi$.
    \item Give an efficient algorithm that transforms a QBF $\Phi$ into an equisatisfiable simple QBF $\Psi$.
    \item Prove the algorithm's correctness.
\end{enumerate}



\section*{Problem 7.}

Outline an interactive proof for TSQBF.
Hint: show that simple formulas have "nice" arithmetizations.

\subsection*{Proof}

$$
TSQBF := \{\Phi | \Phi \text{is a fully quantified boolean formula which is simple and evalautes to true}\}
$$

Let's take an TSQBF problem of the form and arithmetize it:

$$
\forall{x_{1}} \exists{x_{2}} \ \phi(x_{1}, x_{2})
$$

\begin{enumerate}
        \item Replace Boolean variables $x$ with with integer variables
        \item Replace logical operators with arithmetic operators
          \begin{itemize}
            \item[$\circ$] $\lor$ (OR) becomes addition ($+$)
            \item[$\circ$] $\land$ (AND) becomes multiplication ($\times$)
            \item[$\circ$] $\lnot$ (NOT) becomes $(1 - z)$
          \end{itemize}
        \item Replace quantifiers
        \begin{itemize}
            \item[$\circ$] $\forall$ becomes a product ($\prod$) over $\{0,1\}$
            \item[$\circ$] $\exists$ becomes a sum ($\sum$) over $\{0,1\}$
        \end{itemize}
\end{enumerate}

By applying these rules we end up with the resulting arithmetization:

$$
A := \prod_{x_1 \in {0,1}}  \left(\sum_{x_2 \in {0,1}}(x_{1} + x_{2} - x_{1}x_{2})\right)
$$

Which evaluates to the integer $A=1 \cdot 2=2$.

The prover will now attempt to convince the verifer that it knows the claimed
value $a \in A \ (mod \ p)$.  The verifier will send a random element $r_{1} \in F_{p}$ to the prover.  The prover computes the polynomial $f(x_{1}, x_{2}) =
x_{1} + x_{2} - x_{1}x_{2}$, replacing $x_{1}$ with $r_{1}$ and sends $f(r_{1,}, x_{2})$ to the verifier.

The verifer computes the sum over $x_{2}$ for $x_{1} = r_{1}$.

$$
S_{r} = \sum_{x_{2} \in \{0,1\}} f(r_{1,}, x_{2})
$$

Expecting that $A=S_{0} \cdot S_{1}$.

\begin{itemize}
        \item Completeness: If $\Phi$ true and the claimed sum is correct, the protocol will accept with probability 1.
        \item Soundness: If $\Phi$ is false, the protocol will reject with high probability, depending on the field size and the degrees of the polynomials involved.
\end{itemize}


\end{document}
